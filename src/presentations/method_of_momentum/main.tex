% !TEX encoding = UTF-8 Unicode
% !TEX TS-program = xelatex
% !TEX root = main.tex

\documentclass[xetex]{beamer}

%%%%%%%%%%%%%%%%%%%%%%%%%%%%%%%%%%%%%%%%%%%%%%%%%%%%%%%%%%%%%%%%%%%%
% Beamer settings

\mode<presentation>
{
  %\usetheme{Frankfurt}
  \useoutertheme[subsection=false]{smoothbars}
  \setbeamercovered{transparent}
  \setbeamertemplate{navigation symbols}{} %remove navigation symbols
  \setbeamercovered{invisible} %No transparent layers
}


\usepackage[many]{tcolorbox}
%\usepackage{tcolorbox}
\tcbuselibrary{skins,xparse}


\newcounter{ct}
\newcounter{contribution}

% put color to \boxed math command
\newcommand*{\boxcolor}{orange}
\makeatletter
\renewcommand{\boxed}[1]{\textcolor{\boxcolor}{%
\tikz[baseline={([yshift=-1ex]current bounding box.center)}] \node [rectangle, minimum width=1ex,rounded corners,draw] {\normalcolor\m@th$\displaystyle#1$};}}
 \makeatother
 \newcommand{\tboxed}[1]{\textcolor{\boxcolor}{%
\tikz[baseline={([yshift=-1ex]current bounding box.center)}] \node [rectangle, minimum width=1ex,rounded corners,draw] {\normalcolor#1};}}
 \makeatother

\usepackage{etoolbox}

\newtoggle{presentation}
\toggletrue{presentation}
%\togglefalse{presentation}

\newtoggle{euler}
\toggletrue{euler}
%\togglefalse{presentation}
 

% Loads configuration file

\input{./cfg/settings.tex}
\input{./cfg/fonts.tex}
\input{./cfg/commands.tex}
\input{./cfg/environments.tex}

%\usepackage{fixmath}

%%%%%%%%%%%%%%%%%%%%
% BIBLATEX PACKAGE %
%%%%%%%%%%%%%%%%%%%%

% To be loaded locally after the language package

\usepackage[natbib=true,backend=biber,style=alphabetic,sorting=anyvt,
firstinits=true,doi=false,isbn=false,url=false,backref=true,texencoding=utf8,bibencoding=utf8]{biblatex}

\IfFileExists{./../Library.bib}{

 \addbibresource{./../../../Library}
 
}{

  %Must be used with biber as backend
  %Execute with "biber SeparableMetrics"
  \addbibresource[location=remote]{https://www.dropbox.com/s/r00o9lw76cksxvj/Library.bib?dl=1} 
}






\graphicspath{ {./imgs/} }
\newcommand{\svginput}[1]{\input{imgs/#1}}

%%%%%%%%%%%%%
% TITLEPAGE %
%%%%%%%%%%%%%

% Needs to be loaded locally

\usetikzlibrary{shapes,arrows}
\setbeamerfont{author}{size=\LARGE}
\setbeamerfont{institute}{size=\normalsize\itshape}
\setbeamerfont{title}{size=\fontsize{24}{30}\bfseries}
\setbeamerfont{subtitle}{size=\Large\normalfont\slshape}

\setbeamertemplate{title page}{%
\begin{tikzpicture}[remember picture,overlay]
\fill[gblue700]
  ([yshift=0pt]current page.west) rectangle ([yshift=-\headheight] current page.north east);
\node[anchor=east] 
  at ([yshift=85pt]current page.south east) (author)
  {\parbox[t]{\paperwidth}{\centering%
    \usebeamerfont{author}\textcolor{gblue700}{%
    \textpdfrender{
    TextRenderingMode=FillStroke,
    FillColor=gblue700,
    LineWidth=.1ex,
    }{\insertauthor}}}};
%\node[anchor=south east] 
%  at ([yshift=0pt]current page.south east) (institute)
%  {\parbox[t]{.78\paperwidth}{\raggedleft%
%    \usebeamerfont{institute}\textcolor{black}{\insertinstitute}}};
\node[anchor=south] 
  at ([yshift=0pt]current page.south) (logo)
  {\parbox[t]{\paperwidth}{\centering%
    \usebeamercolor[fg]{titlegraphic}\inserttitlegraphic}};
\node[anchor=center]
  at ([yshift=50pt,xshift=0pt]current page.center) (title)
  {\parbox[t]{\textwidth}{\centering%
 \usebeamerfont{author}\textcolor{white}{%
    \textpdfrender{
    TextRenderingMode=FillStroke,
    FillColor=white,
    LineWidth=.1ex,
    }{\inserttitle}}}};
\node[anchor=east]
  at ([yshift=-60pt,xshift=-20pt]current page.east) (subtitle)
  {\parbox[t]{.6\paperwidth}{\raggedleft\usebeamerfont{subtitle}\textcolor{black}{\insertsubtitle}}};
\end{tikzpicture}
}


\title[Method of Momentum]{Method of Momentum}
\author[humberto.steinshiromoto@qbe.com]{Humberto \sc Stein Shiromoto}
\institute{The University of Sydney}
\date[19/05/15]{\today}
\titlegraphic{
\begin{minipage}[b]{0.95\linewidth}
\begin{figure}[htbp!]
\centering
\includegraphics[width=2.5cm]{./imgs/Usyd_new_logo}
\end{figure}
\end{minipage}
}

\begin{document}

\tikzstyle{every picture}+=[remember picture]

% By default all math in TikZ nodes are set in inline mode. Change this to
% displaystyle so that we don't get small fractions.
\everymath{\displaystyle}
	
	{
	%Highlight section in navigation bar
	\setbeamertemplate{section in head/foot shaded}[default][100]
	\setbeamercolor{upper separation line head}{bg=gblue700}
	% Highlight dots
	\setbeamertemplate{mini frame in other subsection}[default][100]
	% Highlight and fill dots
%	\setbeamertemplate{mini frame in other subsection}{%
%    \begin{pgfpicture}{0pt}{0pt}{0.1cm}{0.1cm}%
%        \pgfpathcircle{\pgfpoint{0.05cm}{0.05cm}}{0.05cm}%
%        \pgfusepath{fill,stroke}%
%    \end{pgfpicture}%
%    }%
		
	
	
	\frame{
		\maketitle	
	}	}
	
	\section[Introduction]{Introduction}	
	\subsection[Motivation]{Motivation}	

	\frame[c]{
		\frametitle{Problem Formulation}
		\begin{adjustwidth}{-2em}{-2em}
		Let $D\subset\mathbb{R}^n$ be an open set, and $f\in\mathcal{C}^2(D,\mathbb{R})$

		\begin{equation*}
			\min_{x\in D} f(x)
		\end{equation*}

		\begin{tcolorbox}[title=A Necessary Condition]
		The point $x^\ast\in D$ is a solution to the optimisation problem if
		\begin{equation*}
			\nabla f(x^\ast)=0\;.
		\end{equation*}
		\end{tcolorbox}

		\end{adjustwidth}
	}

	\subsection[Background]{Background}
	\frame[c]{
		\frametitle{Hypothesis}
		\begin{tcolorbox}[title=Assumption on $f$]
		\begin{itemize}
			\item For every $c\in\mathbb{R}_{>0}$, the set
			\[\{x\in\mathbb{R}:f(x)\leq c\}\]
			is closed;

			\item $\inf_{x\in D}f(x)>-\infty$
		\end{itemize}
		\end{tcolorbox}
	}
	% Ref. Boyd, Vanderberghe, Convex optm
	\frame[c]{
		\frametitle{Strong convexity: Definition and Property}

		\begin{itemize}
			\item Definition: The function $f$ is said to be \emph{strong convex} if 
		\[\exists\underline{m}>0, H(\cdot)\succeq mI\;,\]
 		where $H$ is the Hessian of $f$.

 			\item Property\footnote{Borwen and Vanderweff, ``Convex Functions - Constructions, Characterizations and Counterexamples'', CUP, 2010, Ch. 5}
 		\[f(y)\geq f(x)+\nabla f(x)\cdot(y-x)+\dfrac{\underline{m}}{2}|y-x|_2^2\]
 		\end{itemize}

% 		\[H(x)=\nabla^2 f(x)=\begin{bmatrix}
% \dfrac{\partial^2f_1}{\partial x_1^2}&\dfrac{\partial^2f_1}{\partial x_1\partial x_2}&\cdots&\dfrac{\partial^2f_1}{\partial x_1\partial x_n}\\
% \dfrac{\partial^2f_2}{\partial x_2\partial x_1}&\dfrac{\partial^2f_2}{\partial x_2^2}&\cdots&\dfrac{\partial^2f_2}{\partial x_2\partial x_n}\\
% \vdots&\vdots&\ddots&\vdots\\
% \dfrac{\partial^2f_n}{\partial x_n\partial x_1}&\dfrac{\partial^2f_n}{\partial x_n\partial x_2}&\cdots&\dfrac{\partial^2f_n}{\partial x_n^2}
% \end{bmatrix}(x)\]
	}

	\subsection[Gradient Descent]{Gradient Descent}

	\frame[c]{
		\frametitle{Gradient Descent: Algorithm}
		\begin{tcolorbox}
			\begin{enumerate}
				\item Let $k=0$, $\alpha>0$ be the \emph{stepsize}, $\varepsilon_x$, $\varepsilon_f$ be the \emph{tolerances} and $x_0\in D$ be an initial point

				\item Compute
				\[x_{k+1}=x_k - \alpha\nabla f(x_k)\]

				\item If $|x_k - x^\ast|\leq\varepsilon_x$ and/or $|f(x_k) - f(x^\ast)|\leq\varepsilon_f$, stop. Else, go to 2.
			\end{enumerate}
		\end{tcolorbox}
		Think about this
		\begin{tcolorbox}[title=Theorem]
				Let $L$ be the Lipschitz constant of $\nabla f$ and $\alpha< \tfrac{2}{L}$, then,
				\[|f(x_k)-f(x^\ast)|=\dfrac{|x_0-x^\ast|^2}{2\alpha k}\;.\]
		\end{tcolorbox}

		Consequence: let $\varepsilon>0$ be the error tolerance the minimum number of iterations $n$ is $O(1/\varepsilon)$

	}

	\frame[c]{
		Think about this
		% ref: http://users.ece.utexas.edu/~cmcaram/EE381V_2012F/Lecture_4_Scribe_Notes.final.pdf
		\begin{tcolorbox}[title=Lemma]
				If $f$ be Lipschitzian and convex on $D$ with constant $L$, then for every $x,y\in D$,
				 \[f(x)-f(y)-\nabla f(x)\cdot(y-x)\leq\dfrac{L}{2}|x-y|^2\]
		\end{tcolorbox}

		\begin{enumerate}
			% Nesterov, "Introductory lectures on convex optm", Thm 2.1.14
			\item 
			% Note that
			% \begin{align*}
			% 	|x_{k+1}-x^\ast|^2=&\ |x_k-x^\ast-\alpha\nabla f(x_k)|^2\\
			% 	=&\ |x_k-x^\ast|^2-2\alpha\nabla f(x_k)\cdot (x_k-x^\ast)+\alpha^2|\nabla f(x_k)|^2
			% \end{align*} 

			Let $x_{k+1}=x_k-\alpha \nabla f(x_k)$. From the previous lemma, 
			\begin{align*}
			f(x_{k+1})\leq&\ f(x_k) + \nabla f(x_k)(x_{k+1}-x_k)+\dfrac{L}{2}|x_{k+1}-x_k|^2\\
			=&\ f(x_k)- \alpha\left(1-\dfrac{\alpha}{2}L\right)|\nabla f(x_k)|^2
			\end{align*}

			\item Consequently,
			\[|\nabla f(x_k)|^2\leq \dfrac{1}{\alpha\left(1-\dfrac{\alpha}{2}L\right)}(f(x_k)-f(x_{k+1}))\;.\]
		\end{enumerate}
	}

	\frame[c]{
		\begin{enumerate}[3.]
			\item For every $l>0$,
			\[\sum_{k=1}^l |\nabla f(x_k)|^2\leq \dfrac{1}{\alpha\left(1-\dfrac{\alpha}{2}L\right)}(f(x_0)-f(x_l))\]

			\item Thus, $|\nabla f(x_k)|^2\to0$, as $k\to\infty$. Hence $x_k\to x^\ast$;

			\item Also, for every $k>0$,
			\begin{align*}
			f(x_{k+1})-f(x_k)\leq& +\left(\dfrac{L}{2}-\dfrac{1}{\alpha}\right)|x_{k+1}-x_k|^2
			\end{align*}
		\end{enumerate}
	}


	\section[Method of Momentum]{Method of Momentum}
	\subsection{Introduction}
	\frame[c]{
		\frametitle{A}
		Add a memory term to the gradient descent algorithm
		\begin{equation*}
			\left\{\begin{array}{rcl}
			z^+&=&\beta z+\nabla f(x)\\
			x^+&=&x-\alpha z^+\;.
			\end{array}
			\right.
		\end{equation*}
	}

	\frame[c]{
		\frametitle{Dynamics of momentum}
		\begin{adjustwidth}{-2em}{-2em}
		Let
		\[f(x)=\dfrac{1}{2}x^\top Ax-b^\top x\,\]
		where $A$ is symmetric and invertible.

		\begin{equation*}
			\left\{\begin{array}{rcl}
			z^+&=&\beta z+(Ax-b)\\
			x^+&=&x-\alpha z^+\;.
			\end{array}
			\right.
		\end{equation*}

		Let $Q$ be such that $A=Q\Lambda Q^\top$. 
		Changing coordinates, $w=Q(x-x^\ast)$ and $y=Qz$
		\begin{equation*}
			\left\{\begin{array}{rcl}
			y_i^+&=&\beta y_i+\lambda_i w_i\\
			w_i^+&=&-\alpha \beta y_i+(1-\alpha\lambda_i)w_i\;.
			\end{array}
			\right.
		\end{equation*}

		This equation can be written as $v^+=Rv$.
	\end{adjustwidth}
	}

	\frame[c]{
		\frametitle{Linear Difference Systems}
	\begin{adjustwidth}{-2em}{-2em}
		The convergence depends on the solutions to the system $v^+=Rv$

		The analysis of the matrix $R$ provides the behaviour of the 
		convergence of the method

		A solution to $v^+=Rv$, starting from an \emph{initial condition}
		$v_0$, is function $V:\mathbb{R}_{\geq0}\to\mathbb{R}^n$ defined as
		$V(k,v)$
	\end{adjustwidth}
	}

	\subsection{Items and lists}
	\frame[c]{
		\frametitle{The second frame: items}
		\begin{adjustwidth}{-2em}{-2em}
		\blindlist{itemize}[3]	
		\end{adjustwidth}
	}

	\frame[c]{
		\frametitle{The third frame: numbered items}
		\begin{adjustwidth}{-2em}{-2em}
		\blindlist{enumerate}[5]	
		\end{adjustwidth}
	}
	
	\frame[c]{
		\frametitle{The fourth frame: description list}
		\begin{adjustwidth}{-2em}{-2em}
		\blindlist{description}[2]	
		\end{adjustwidth}
	}
	
	\subsection{Figure}
	\frame[c]{
		\frametitle{The Second Frame: a figure}
		\begin{adjustwidth}{-2em}{-2em}
		\begin{figure}[htpb!]
			\centering
			\scalebox{0.75}{\input{./imgs/set_evolution.eps_tex}}
		\end{figure}
		\end{adjustwidth}
		}
	\section{Tests with boxes}
	\subsection{Box without shadows}
	\frame[c]{
		\frametitle{Without shadows}
		
		\begin{tcolorbox}[title=A test]
This is an equation
\begin{equation*}
	\int_{X(t,\mathbf{Z})}\rho\,dx-\int_\mathbf{Z}=\int_0^t\int_{X(s,\mathbf{Z})}\nabla(\rho f)(x)\,dx
\end{equation*}
\end{tcolorbox}\par\bigskip
\begin{tcolorbox}
This is a tcolorbox.
\end{tcolorbox}\par\bigskip
}

	\subsection{Lifted shadows}
	
	\frame[c]{
	\frametitle{Lifted shadows}
				\begin{tcolorbox}[drop lifted shadow]
This is a tcolorbox.
\end{tcolorbox}\par\bigskip
\begin{tcolorbox}[title=Another shadow,
drop lifted shadow]
This is a tcolorbox.
\end{tcolorbox}\par\bigskip
\begin{tcolorbox}[drop lifted shadow]
\begin{equation*}
% Todo: check why \dot{x} doesn't work
	\dfrac{dx}{dt}=f(x)
\end{equation*}
\end{tcolorbox}\par\bigskip
\begin{tcolorbox}[code={\pgfkeysalsofrom\redblock},title=A test,drop lifted shadow]
Test.
\end{tcolorbox}		
	}
	
	\subsection{Midday Shadows}
	
	\frame[c]{
		\frametitle{Midday shadows}
		
		\begin{tcolorbox}[drop fuzzy midday shadow,
enhanced,colback=red!5!white,colframe=gred900]
This is a tcolorbox.
\end{tcolorbox}\par\bigskip
\begin{tcolorbox}[title=Another shadow,drop fuzzy midday shadow,
enhanced,colback=red!5!white,colframe=gred900]
This is a tcolorbox.
\end{tcolorbox}
}

% \frame[t]{
% 	% Requires definition of \normal and \light fonts
% 	\frametitle{Variations on the title: available if \texttt{normal} and \texttt{light} defined}
% 	\begin{tcolorbox}[title=Another shadow,
% drop lifted shadow]
% This is a tcolorbox.
% \end{tcolorbox}\par\bigskip
% \begin{tcolorbox}[fonttitle=\normal,title=Another shadow,
% drop lifted shadow]
% This is a tcolorbox.
% \end{tcolorbox}\par\bigskip
% \begin{tcolorbox}[fonttitle=\light,title=Another shadow,
% drop lifted shadow]
% This is a tcolorbox.
% \end{tcolorbox}
% }

			\frame[c]{
	\frametitle{Small boxes with midday, and lifted shadows}
	\begin{minipage}[c]{0.30\linewidth}
	\begin{tcolorbox}[code={\pgfkeysalsofrom\postitblock},drop fuzzy midday shadow,hbox]
Test.
\end{tcolorbox}
\begin{tcolorbox}[code={\pgfkeysalsofrom\whiteblock},drop fuzzy midday shadow,hbox]
Test.
\end{tcolorbox}
	\end{minipage}
	\begin{minipage}[c]{0.30\linewidth}
	\begin{tcolorbox}[code={\pgfkeysalsofrom\postitblock},drop lifted shadow,hbox]
Test.
\end{tcolorbox}
\begin{tcolorbox}[code={\pgfkeysalsofrom\whiteblock},drop lifted shadow,hbox]
Test.
\end{tcolorbox}
	\end{minipage}
	\begin{minipage}[c]{0.30\linewidth}
	\begin{tcolorbox}[code={\pgfkeysalsofrom\postitblock},drop fuzzy shadow south,hbox]
Test.
\end{tcolorbox}
\begin{tcolorbox}[code={\pgfkeysalsofrom\whiteblock},drop fuzzy shadow south,hbox]
Test.
\end{tcolorbox}
	\end{minipage}
	
	\begin{tcolorbox}[code={\pgfkeysalsofrom\postitblock},drop lifted shadow]
	\begin{varwidth}{\textwidth}
	The text
	
       $\int_{X(t,\mathbf{Z})}\rho\,dx-\int_\mathbf{Z}=\int_0^t\int_{X(s,\mathbf{Z})}\nabla(\rho f)(x)\,dx$
       \end{varwidth}
       \end{tcolorbox}
	}
	
	\section{Theorem-like environments}
	\subsection{Theorem environments}
	\frame[c]{	
		\frametitle{ToDo}
	
	}
	\subsection{Definition environments}
	
	\subsection{Example environments}
		
\end{document}
