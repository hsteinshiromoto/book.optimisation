% !TEX encoding = UTF-8 Unicode
% !TEX root = ../main.tex

\chapter{Variational Analysis}\label{sec:VA}

To formulate the variational problem, first consider the $\mathcal{L}_p$
norm defined as
\begin{equation}\label{eq:lp norm}
	||y||_{\mathcal{L}_p}=\left(\int_a^b |y(x)|^p\,dx\right)^{1/p},
\end{equation}
where $p\in\mathbb{R}_{\geq0}$. Now, the local minima of function can be
defined

\begin{definition}\label{def:VA:local minima}
	Let $\mathbf{V}$ be a vector space of functions equipped with a norm 
	$||\cdot||$, let $\mathbf{A}$ be a subset of $\mathbf{V}$, and let 
	$J$ be a real-valued functional defined on $\mathbf{A}$. A function 
	$y^\ast\in \mathbf{A}$ is a \emph{local minimum} of $J$ over 
	$\mathbf{A}$ if there exists $\varepsilon>0$ such that the inequality
	$$J(y^\ast)\leq J(y)$$
	holds, for all $y\in \mathbf{A}$ satisfying $||y-y^\ast||<\varepsilon$.
\end{definition}

\section{First Variation and First-order Necessary Condition}

\begin{definition}\label{def:VA:first variation}[First variation]
	Let $\mathbf{V}$ be a function space, and let 
	$J:\mathbf{V}\to\mathbb{R}$ be a functional. The linear functional
	$\delta J|_y:\mathbf{V}\to\mathbb{R}$ is called the \emph{first 
	variational of $J$ at $y$} if, for every $\eta\in\mathbf{V}$ and for
	every $\alpha\in\mathbb{R}$, the equality
	\begin{equation}\label{eq:VA:first variational expansion}
		J(y+\alpha\eta) = J(y) + \delta J|_y(\eta)\alpha + o(\alpha)\;,
	\end{equation}
	where $o$ represents higher-order terms\footnote{$o(\alpha)/\alpha\to0$, 
	as $\alpha\to0$.}.
\end{definition}

The first variation defined in Definition~\ref{def:VA:first variation} 
corresponds to the Gateaux derivative of $J$:
\[\delta J|_y(\eta)=\lim_{\alpha\to0}\dfrac{J(y+\alpha\eta)-J(y)}{\alpha}\;.\]

\begin{definition}
	Let $\mathbf{V}$ be a function space, and let 
	$J:\mathbf{V}\to\mathbb{R}$ be a functional. Let $y^\ast$ be a local
	minimum of $J$, the function $\alpha\eta$, where $\alpha\in\mathbb{R}$
	and the function $\eta\in\mathbf{V}$, is said to be an 
	\emph{admissible perturbation} (with respect to the subset 
	$\mathbf{A}$) if $y^\ast+\alpha\eta\in\mathbf{A}$.
\end{definition}

\begin{theorem}\label{thm:VA:first order necessary condition}
	Let $\mathbf{V}$ be a function space, and let 
	$J:\mathbf{V}\to\mathbb{R}$ be a functional. If $y^\ast\in\mathbf{A}$
	is a local minimum of $J$ over $\mathbf{A}$, then
	\[\delta J|_{y^\ast}(\eta)=0\;.\]
\end{theorem}

\begin{proof}
	Theorem~\ref{thm:VA:first order necessary condition} claims that
	\[\delta J|_{y^\ast}(\eta)=0\;.\]

	To show this, suppose that $\delta J|_{y^\ast}(\eta)\neq0$. Then,
	from the higher-order term of equation~\eqref{eq:VA:first variational expansion},
	there exists $\varepsilon>0$ small enough so that the inequality
	\[||\alpha||<\varepsilon, \alpha\neq0\]
	implies
	\[||o(\alpha)||<||\delta J|_{y^\ast}(\eta)\alpha||\;.\]

	For these values of $\alpha$, equation~\eqref{eq:VA:first variational expansion}
	gives
	\begin{equation*}
		J(y^\ast+\alpha\eta)-J(y^\ast)<\delta J|_{y^\ast}(\eta)\alpha + 
		||\delta J|_{y^\ast}(\eta)\alpha||\;.
	\end{equation*}
	If $\alpha$ is restricted to have the opposite sign to 
	$\delta J|_{y^\ast}(\eta)$, the the previous equation becomes
	$J(y^\ast+\alpha\eta)-J(y^\ast)<0$. But this contradicts the fact
	that $J$ has a minimum at $y^\ast$. Thus, the conclusion holds.
\end{proof}